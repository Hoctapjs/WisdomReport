% bắt đầu phan khai báo, cấu hình gói và định dạng tài liệu
  % Nó dùng để định nghĩa loại tài liệu và nạp các "gói" (packages) để mở rộng chức năng.
  
  % Dòng này khai báo đây là một bài báo và sẽ được hiển thị thành 2 cột
  \documentclass[twocolumn]{article}

  % mỗi dòng tạo 1 gói có chức năng riêng
  \usepackage[utf8]{inputenc}
  \usepackage[T1]{fontenc}

  % thiết lập lề của trang là 1.25 inch
  \usepackage[margin=1.25in]{geometry}
  \usepackage{authblk}
  \usepackage{setspace}

  % chèn hình ảnh vào tài liệu
  \usepackage{graphicx}
  \usepackage{subcaption}

  % hỗ trợ viết công thức toán học phức tạp
  \usepackage{amsmath}
  \usepackage{booktabs}
  \usepackage{array}
  \usepackage{xcolor}

  % Gói để trình bày code đẹp hơn
  \usepackage{listings} 
  \usepackage{hyperref}

  % Cấu hình hiển thị code
  \lstset{
      basicstyle=\ttfamily\small,
      backgroundcolor=\color{gray!10},
      frame=single,
      breaklines=true,
      captionpos=b
  }

  \setlength{\columnsep}{24pt}
  \usepackage[switch]{lineno}
  \setlength\linenumbersep{8pt}

  %%%%%% Bibliography %%%%%%
  % Sử dụng gói biblatex với kiểu trích dẫn NEJM và sắp xếp theo thứ tự xuất hiện, danh mục tài liệu tham khảo (bibliography) từ tệp sample.bib
  \usepackage[style=nejm, citestyle=numeric-comp, sorting=none]{biblatex}
  \addbibresource{sample.bib} 

  %%%%%% Title & Authors %%%%%%
  % Định nghĩa tiêu đề của bài báo
  \title{\textbf{Full Implementation of BaSyx Asset Administration Shell Infrastructure: From Web UI to Real-Time Data Integration}}
  
  % Định nghĩa tác giả và thông tin liên quan
  \author[1*]{Nam - Phong - Son}

  % Định nghĩa thông tin tác giả và thời gian báo cáo
  \affil[1]{Reporter, System Integration Team}
  \affil[*]{Reporting Period: January 07, 2026 -- January 13, 2026}

  \setstretch{1.15}

% end phần khai báo, bắt đầu phần nội dung

% bắt đầu phần thông tin và nội dung chính của tài liệu

  \begin{document}

  \twocolumn[
    \begin{@twocolumnfalse}
      \maketitle % tạo tiêu đề bài báo

      % phần tóm tắt nội dung nghiên cứu
      \begin{abstract}
        This comprehensive study details the deployment and configuration of an industrial Digital Twin platform utilizing the Eclipse BaSyx framework. The research covers the end-to-end implementation including the React-based Web UI, a Dockerized AAS environment, and a persistent storage mechanism using MongoDB Atlas. We investigate the integration of a BaSyx DataBridge to facilitate real-time telemetry from IoT devices using MQTT and JSONata transformations. Results demonstrate a functional three-column visualization system and a robust ETL workflow capable of managing complex asset lifecycles in Industry 4.0 environments.
      \end{abstract}
      \vspace{1.5em}
    \end{@twocolumnfalse}
  ]

  % Tạo phần giới thiệu Introduction
  \section{Introduction}
  The development of a Digital Twin (DT) requires a standardized method to represent physical assets in a digital format \cite{ref169}. This project implements the Asset Administration Shell (AAS) model following the IEC 63278 standard to ensure interoperability \cite{ref170, ref171}. The primary goal is to establish a secure, reliable information exchange layer using the BaSyx AAS Environment \cite{ref4, ref5}. 

  A critical challenge in standard AAS architectures is the volatile nature of in-memory storage, which risks losing historical states during system restarts \cite{ref76}. To address this, we integrate MongoDB as a persistent NoSQL layer \cite{ref78}. Furthermore, the system incorporates middleware for Extract-Transform-Load (ETL) operations to bridge the gap between Operational Technology (OT) and Information Technology (IT) \cite{ref107}.

  \section{Materials and Methods}

  \subsection{Web UI and Frontend Setup}
  The frontend utilizes the official Eclipse BaSyx web-ui repository \cite{ref12}. Deployment involves:
  
  % tạo danh sách gạch đầu dòng (bullet points) để liệt kê các bước triển khai
  \begin{itemize}
      \item Cloning the repository and installing dependencies via \texttt{npm install} \cite{ref15}.
      \item Executing \texttt{npm run dev} to initialize the interface at \texttt{http://localhost:3000}.
      \item Configuring \texttt{basyx-infra.yml} to point to the backend service at port 8081.
  \end{itemize}

  \subsection{Backend and Database Integration}
  The backend leverages a Docker Compose environment \cite{ref21}. To enable persistence, the Java Server is configured with the \texttt{mongoDbStorage} profile \cite{ref84}.

  % tạo 1 khổi để hiển thị code bash với chú thích
  \begin{lstlisting}[language=bash, caption=Docker Compose environment configuration]
  services:
    aas-env:
      image: eclipsebasyx/aas-environment:2.0.0
      ports: ["8081:8081"]
      environment:
        - SPRING_PROFILES_ACTIVE=mongoDbStorage
        - SPRING_DATA_MONGODB_DATABASE=DT_DB
        - Basyx_Cors_Allowed-Origins=*
  \end{lstlisting}

  % tạo một đoạn văn giải thích về cấu hình Docker Compose
  \subsection{Real-Time Data Routing}
  The BaSyx DataBridge connects MQTT telemetry to the AAS server \cite{ref107}. We configured a \texttt{routes.json} file to map CPU and RAM usage metrics from a Python simulation script to specific Submodel elements. The workflow uses JSONata queries to extract values and HTTP PATCH requests for updates \cite{ref111}.

  % tạo phần results để trình bày kết quả của quá trình triển khai
  \section{Results}

  % tạo phần con để trình bày các tính năng của giao diện người dùng và luồng hoạt động
  \subsection{UI Features and Flow}
  The interface employs a three-column layout:

  % tạo danh sách gạch đầu dòng để mô tả các phần của giao diện người dùng
  \begin{itemize}
      \item \textbf{Left:} AAS list with search and upload (\texttt{.aasx}).
      \item \textbf{Middle:} Hierarchical Submodel tree visualization.
      \item \textbf{Right:} Detailed visualizations and JSON raw data views.
  \end{itemize}

  % tạo phần con để trình bày về lưu trữ dữ liệu và kiến trúc API
  \subsection{Persistent Storage Results}
  Data is automatically organized into MongoDB collections: \texttt{assetAdministrationShells}, \texttt{submodels}, and \texttt{conceptDescriptions} \cite{ref100}. This ensures data persistence across container restarts \cite{ref78}.

  % tạo phần con để trình bày về kiến trúc API và các điểm cuối (endpoints) quan trọng
  \subsection{API Architecture Reference}
  The system maps functions according to the ISO 23247-2 model. Key schema attributes are summarized in Table \ref{tab:schema}.

  % tạo một bảng để trình bày các thuộc tính chính của schema và vai trò kỹ thuật của chúng
  \begin{table}[h]
      \caption{Key Schema Attributes and Technical Roles}    
      \centering
      \small
      \begin{tabular}{lp{1.8cm}p{3cm}}
              \toprule
              Schema & Attribute & Purpose \\
              \midrule
              AAS & \texttt{submodels} & Links Shell to functional sets. \\
              Submodel & \texttt{elements} & Stores states and specs. \\
              Result & \texttt{messages} & Provides status feedback. \\
              Reference & \texttt{keys} & Defines retrieval paths. \\
              \bottomrule
      \end{tabular}
      \label{tab:schema}
  \end{table}

  % tạo một bảng khác để trình bày các endpoint API quan trọng và chức năng của chúng
  \begin{table*}[t]
      \caption{Comprehensive API Endpoint Reference for AAS Environment \cite{ref229}}    
      \centering
      \small
      \begin{tabular}{llll}
              \toprule
              Functional Group & Method & Endpoint & Description \\
              \midrule
              AAS Repository & GET & \texttt{/shells} & Lists all shells. \\
              AAS Repository & POST & \texttt{/shells} & Registers a new shell. \\
              Submodel Repo & GET & \texttt{/submodels/\{id\}} & Returns submodel data. \\
              Submodel Repo & PATCH & \texttt{/submodels/\{id\}/\$value} & Updates values only. \\
              Operation & POST & \texttt{/.../invoke} & Triggers functions. \\
              File/Attachment & PUT & \texttt{/.../attachment} & Uploads file content. \\
              \bottomrule
      \end{tabular}
      \label{tab:endpoints}
  \end{table*}

  % tạo phần thảo luận để trình bày về những khó khăn gặp phải và cách giải quyết chúng, cũng như những điểm cần cải thiện trong tương lai
  \section{Discussion}
  Implementation encountered 404 and CORS errors due to infrastructure misalignments \cite{ref59}. These were resolved by synchronizing the UI configuration and enabling global CORS origins.

  The use of Value-Only representation (\texttt{/\$value}) significantly reduced network overhead. While the current setup uses Docker for demonstration, production should transition to the full Java SDK for scalability \cite{ref191}.

  % tạo phần kết luận để tóm tắt những điểm chính của nghiên cứu và hướng phát triển trong tương lai
  \section*{Acknowledgments}
  Special thanks to the system integration team for their support in configuring the AAS environment \cite{ref9}.

  % tạo phần tài liệu tham khảo để liệt kê các nguồn đã được trích dẫn trong bài báo
  \printbibliography

  \end{document}

% end phần thông tin và nội dung chính của tài liệu